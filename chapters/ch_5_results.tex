\chapter{Results and Discussions}
In this section we will compare the results our smoothing algorithms with one other with and without flipping for 2D and 3D. First we present a comparison of smoothing algorithms in terms of speed, quality and grading of the meshes on the scale of Poor < Medium < Good. Then We will show the results based on distribution of min angles and radius ratios. First on the left are mesh output, middle figures are distribution of min angles and right figures are distribution of radius ratios. All iteration smoothing are performed on interior vertices. Handling of boundary vertices need extra care and need extension in algorithms \cite{Erten_2009} to accommodate that.

\section{Smoothing on 2D mesh}
Figure \ref{fig:2d} shows smoothing algorithms with flipping. The square contains approximately 300 points and original figure \ref{fig:original_2d} shows the unsmoothed delaunay triangulation. It's a bad triangulation to start with but effectiveness of algorithms can be seen from mesh output. 

\begin{table}[ht]
\caption{Qualitative comparison of smoothing methods in 2D}
\begin{center}
\begin{tabular}{ |c|c|c|c|c| } 
\hline
Smoothing method & Ease of Implementation & Speed & Quality & Grading \\ 
\hline
Smart Laplacian & Good & Good & Medium & Good \\ 
\hline
Optimal Delaunay Triangulation based & Medium & Good & Good & Poor \\ 
\hline
Aspect ratio based & Medium & Medium & Good & Poor  \\ 
\hline
Average circumcenter & Good & Medium & Good & Good  \\
\hline
\end{tabular}
\end{center}
\end{table}

\begin{figure}[H]
    \centering
    \begin{subfigure}[b]{1.0\textwidth}
        \includegraphics[width=0.32\textwidth]{images/2d/original_m.pdf}
        \includegraphics[width=0.32\textwidth]{images/2d/original_min_angle.pdf}
        \includegraphics[width=0.32\textwidth]{images/2d/original_radius_ratio.pdf}
        \caption{Origianl}       
        \label{fig:original_2d}  
    \end{subfigure}
\end{figure}

\begin{figure}[H]
	\ContinuedFloat
    \centering
    \begin{subfigure}[b]{1.0\textwidth}
        \includegraphics[width=0.32\textwidth]{images/2d/laplacian_m.pdf}
        \includegraphics[width=0.32\textwidth]{images/2d/laplacian_min_angle.pdf}
        \includegraphics[width=0.32\textwidth]{images/2d/laplacian_radius_ratio.pdf}
        \caption{Smart laplacian}       
        \label{fig:avg_sl_2d}        
    \end{subfigure}
\end{figure}

\begin{figure}[H]
	\ContinuedFloat
    \centering
    \begin{subfigure}[b]{1.0\textwidth}
        \includegraphics[width=0.32\textwidth]{images/2d/odt_m.pdf}
        \includegraphics[width=0.32\textwidth]{images/2d/odt_min_angle.pdf}
        \includegraphics[width=0.32\textwidth]{images/2d/odt_radius_ratio.pdf}
        \caption{Optimal delaunay triangulation}       
        \label{fig:odt_2d}        
    \end{subfigure}
\end{figure}

\begin{figure}[H]
	\ContinuedFloat
    \centering
    \begin{subfigure}[b]{1.0\textwidth}
        \includegraphics[width=0.32\textwidth]{images/2d/aspect_ratio_m.pdf}
        \includegraphics[width=0.32\textwidth]{images/2d/aspect_ratio_min_angle.pdf}
        \includegraphics[width=0.32\textwidth]{images/2d/aspect_ratio_radius_ratio.pdf}
        \caption{Aspect ratio based}       
        \label{fig:avg_ar_2d}
    \end{subfigure} 
\end{figure}

\begin{figure}[H]
	\ContinuedFloat
    \centering
    \begin{subfigure}[b]{1.0\textwidth}
        \includegraphics[width=0.32\textwidth]{images/2d/average_circumcenter_m.pdf}
        \includegraphics[width=0.32\textwidth]{images/2d/average_circumcenter_min_angle.pdf}
        \includegraphics[width=0.32\textwidth]{images/2d/average_circumcenter_radius_ratio.pdf}
        \caption{Average circumcenter}       
        \label{fig:avg_cc_2d}
    \end{subfigure}
    \caption{Smoothing result on 2D mesh}
    \label{fig:2d}
    \vspace{0.5 cm}
\end{figure}

\section{Smoothing on 3D mesh}
Figure \ref{fig:3d} shows smoothing algorithms with flipping in 3D. The mesh shown is of cylinder with approximately 1000 vertices, most of them are on the surface. Original figure \ref{fig:original_3d} shows the unsmoothed delaunay tetrahedralization. It's a refined mesh to start with but still smoothing improves the overall quality of the mesh.

\begin{table}[ht]
\caption{Qualitative comparison of smoothing methods in 3D}
\begin{center}
\begin{tabular}{ |c|c|c|c|c| } 
\hline
Smoothing method & Ease of Implementation & Speed & Quality & Grading \\ 
\hline
Smart Laplacian & Good & Good & Poor & Poor \\ 
\hline
Optimal Delaunay Triangulation based & Medium & Medium & Good & Poor \\ 
\hline
Aspect ratio based & Medium & Medium & Good & Poor  \\ 
\hline
Average circumcenter & Good & Good & Good & Good  \\
\hline
\end{tabular}
\end{center}
\end{table}

\begin{figure}[H]
    \centering
    \begin{subfigure}[b]{1.0\textwidth}
        \includegraphics[width=0.32\textwidth]{images/3d/original_m.pdf}
        \includegraphics[width=0.32\textwidth]{images/3d/original_min_angle.pdf}
        \includegraphics[width=0.32\textwidth]{images/3d/original_radius_ratio.pdf}
        \caption{Origianl}       
        \label{fig:original_3d}  
    \end{subfigure}
\end{figure}

\begin{figure}[H]
	\ContinuedFloat
    \centering
    \begin{subfigure}[b]{1.0\textwidth}
        \includegraphics[width=0.32\textwidth]{images/3d/laplacian_m.pdf}
        \includegraphics[width=0.32\textwidth]{images/3d/laplacian_min_angle.pdf}
        \includegraphics[width=0.32\textwidth]{images/3d/laplacian_radius_ratio.pdf}
        \caption{Smart laplacian}       
        \label{fig:avg_sl_3d}        
    \end{subfigure}
\end{figure}

\begin{figure}[H]
	\ContinuedFloat
    \centering
    \begin{subfigure}[b]{1.0\textwidth}
        \includegraphics[width=0.32\textwidth]{images/3d/odt_m.pdf}
        \includegraphics[width=0.32\textwidth]{images/3d/odt_min_angle.pdf}
        \includegraphics[width=0.32\textwidth]{images/3d/odt_radius_ratio.pdf}
        \caption{Optimal delaunay triangulation}       
        \label{fig:odt_3d}        
    \end{subfigure}
\end{figure}

\begin{figure}[H]
	\ContinuedFloat
    \centering
    \begin{subfigure}[b]{1.0\textwidth}
        \includegraphics[width=0.32\textwidth]{images/3d/aspect_ratio_m.pdf}
        \includegraphics[width=0.32\textwidth]{images/3d/aspect_ratio_min_angle.pdf}
        \includegraphics[width=0.32\textwidth]{images/3d/aspect_ratio_radius_ratio.pdf}
        \caption{Aspect ratio based}       
        \label{fig:avg_ar_3d}
    \end{subfigure} 
\end{figure}

\begin{figure}[H]
	\ContinuedFloat
    \centering
    \begin{subfigure}[b]{1.0\textwidth}
        \includegraphics[width=0.32\textwidth]{images/3d/average_circumcenter_m.pdf}
        \includegraphics[width=0.32\textwidth]{images/3d/average_circumcenter_min_angle.pdf}
        \includegraphics[width=0.32\textwidth]{images/3d/average_circumcenter_radius_ratio.pdf}
        \caption{Average circumcenter}       
        \label{fig:avg_cc_3d}
    \end{subfigure}
    \caption{Smoothing results on 3D mesh}
    \label{fig:3d}
    \vspace{0.5 cm}
\end{figure}

\section{Implementation Details}
We implemented incremental delaunay triangulation for 2D and 3D using flips in C++. For 2D, we used half edge data structure and it can robustly handle upto million points.
For 3D, we used tetrahedron based data structure based on Tetgen \cite{tetgen}. For robustness, we used Jonathan Schewchuck's code for robust predicates which is free for academic purposes. We used 2.2 GHz machine with 8 GB RAM to generate our results. We used extensively Paraview based on VTK geometric library for generating  snapshots and graphs for distribution of aspect ratios and min angles.